\chapter{Présentation des modules}
\label{chap:presentation-modules}

\section{Module \texttt{is\_valid}}

\begin{listing}[!htpb]
    \begin{minted}{c}
        bool is_valid(int pos, action a, levelinfo level, levelinfo air_level);
    \end{minted}
    \caption{Prototype de \texttt{is\_valid} en C.}
    \label{listing:c-is_valid-prototype}
\end{listing}

\subsection{Description}

Dans l'algorithme A*, on a besoin de récupérer tous les voisins d'une case donnée.
On appelle voisin d'une case une case atteignable en un seul mouvement.
\newline\newline
La fonction \texttt{is\_valid} prend donc en paramètre une action, et retourne un booléen indiquant si cette action est valide.

\subsection{Paramètres}

\begin{table}[!htpb]
    \label{tab:parameters-is_valid}
    \begin{tabularx}{\textwidth}{lXX}
        \toprule
        \textbf{Paramètre} & \textbf{Type} & \textbf{Description} \\
        \midrule
        \texttt{pos} & \texttt{int} & Position de la case à tester. \\
        \texttt{a} & \texttt{action} & Action à tester. \\
        \texttt{level} & \texttt{levelinfo} & Structure contenant les informations du niveau. \\
        \texttt{air\_level} & \texttt{levelinfo} & Structure contenant les informations du niveau sans les ennemis. \\
        \bottomrule
    \end{tabularx}
    \caption{Paramètres de la fonction \texttt{is\_valid}.}
\end{table}

\subsection{Choix d'Implémentation}

On choisit d'utiliser un switch sur l'action à tester puisque chaque action a des conditions différentes.

\subsection{Pseudo-code}

\begin{longlisting}
    \begin{minted}{text}
        Fonction is_valid en booléen
        Parametres :
            pos en entier
            a en action
            level en levelinfo
            air_level en levelinfo
        Declarations :
            x en entier
            y en entier
            map en tableau de caractères
            air_map en tableau de caractères
            not_in_air en booléen
        Debut
            x <- pos % level.xsize // On récupère la coordonnée x de la case
            y <- pos / level.xsize // On récupère la coordonnée y de la case
            map <- level.map // On récupère la carte du niveau
            air_map <- air_level.map // On récupère la carte du niveau sans les ennemis

            // Variable indiquant si le joueur n'est pas en l'air
            not_in_air <- (air_map[y + 1][x] != PATH && air_map[y + 1][x] != CABLE && air_map[y + 1][x] != BOMB) || air_map[y - 1][x] == CABLE
            // Si la case en dessous du joueur n'est pas un chemin, un cable ou une bombe, ou si la case au dessus du joueur est un cable, alors le joueur n'est pas en l'air

            Selon a faire
                Cas NONE :
                    Retourner VRAI
                Cas UP :
                    // On ne peut monter que si on est sur une echelle et qu'il n'y a pas de mur au dessus
                    Si map[y][x] == LADDER et map[y - 1][x] != WALL et map[y - 1][x] != FLOOR et map[y - 1][x] != ENEMY alors
                        Retourner VRAI
                Cas DOWN :
                    // On ne peut descendre que si il y a une echelle, un chemin ou un cable en dessous
                    // On laisse la possibilite de descendre si le joueur est en l'air
                    Si (map[y + 1][x] == LADDER ou map[y + 1][x] == PATH ou map[y + 1][x] == CABLE) et map[y + 1][x] != ENEMY alors
                        Retourner VRAI
                Cas LEFT :
                    // On ne peut aller a gauche que si il n'y a pas de mur a gauche et que le joueur n'est pas en l'air
                    // C'est un petit hack car le moteur avance de plusieurs tour de jeu sans utiliser le code du joueur tant qu'il tombe, mais le A* ne le sait pas,
                    // alors on se debrouille pour que la seule action possible soit DOWN
                    Si map[y][x - 1] != WALL et map[y][x - 1] != FLOOR et map[y][x - 1] != ENEMY et map[y][x - 1] != DEAD et map[y + 1][x - 1] != ENEMY et not_in_air alors
                        Retourner VRAI
                Cas RIGHT :
                    // De meme pour la droite
                    Si map[y][x + 1] != WALL et map[y][x + 1] != FLOOR et map[y][x + 1] != ENEMY et map[y][x + 1] != DEAD et map[y + 1][x + 1] != ENEMY et not_in_air alors
                        Retourner VRAI
                Defaut :
                    Afficher "ERROR: Invalid action"
                    Sortir du programme
            Fin Selon

            Retourner FAUX
        Fin
    \end{minted}
    \caption{Pseudo-code de la fonction \texttt{is\_valid}.}
    \label{listing:c-is_valid}
\end{longlisting}

\newpage

\section{Module \texttt{get\_closest\_bonus}}

\begin{listing}[!htpb]
    \begin{minted}{c}
        bonus_list get_closest_bonus(bonus_list bonusl, character_list runner, bonus_list already_seen);
    \end{minted}
    \caption{Prototype de \texttt{get\_closest\_bonus} en C.}
    \label{listing:c-get_closest_bonus-prototype}
\end{listing}

\subsection{Description}

A*, nécessite un point de départ et un point d'arrivée.
Dans notre cas, le point de départ est le runner, et les points d'arrivée sont les bonus.
Mais, afin de finir le niveau plus rapidement (et pour éviter des cas de boucles infinies), on cherche à se rendre sur le bonus le plus proche du runner.
\newline
De plus, si le bonus le plus proche est inaccessible pour une raison quelconque, on passe aux deuxième plus proche, et ainsi de suite.
C'est pourquoi on a besoin de la liste des bonus déjà traités, pour ne pas les traiter une seconde fois.

\subsection{Paramètres}

\begin{table}[!htpb]
    \begin{tabularx}{\textwidth}{lXX}
        \toprule
        \textbf{Paramètre} & \textbf{Type} & \textbf{Description} \\
        \midrule
        \texttt{bonusl} & \texttt{bonus\_list} & Liste des bonus. \\
        \texttt{runner} & \texttt{character\_list} & Liste des personnages. \\
        \texttt{already\_seen} & \texttt{bonus\_list} & Liste des bonus déjà vus. \\
        \bottomrule
    \end{tabularx}
    \caption{Paramètres de la fonction \texttt{get\_closest\_bonus}.}
    \label{tab:parameters-get_closest_bonus}
\end{table}

\subsection{Choix d'Implémentation}

Les listes de bonus sont des listes chaînées, on choisit donc une boucle \texttt{Tant que} pour parcourir la liste des bonus.
La fonction \texttt{is\_in\_bonus\_list} fonctionne similairement, on ne la détaille donc pas.

\newpage

\subsection{Pseudo-code}

\begin{listing}[!htpb]
    \begin{minted}{text}
        Fonction get_closest_bonus en bonus_list
        Parametres :
            bonusl en bonus_list
            runner en character_list
            already_seen en bonus_list
        Declarations :
            closest_bonus en bonus_list
            best_dist en reel
            current en bonus_list
        Debut
            closest_bonus <- NULL

            Si bonusl == NULL alors
                // Si la liste des bonus est vide, il n'y a pas de bonus le plus proche
                Retourner NULL
            Fin Si

            best_dist <- 100000 // On initialise la meilleure distance à une valeur très grande
            current <- bonusl // On initialise le bonus courant à la tête de la liste des bonus

            Tant que current != NULL faire
                // On parcourt la liste des bonus
                Si dist(current->b.x, current->b.y, runner->c.x, runner->c.y) < best_dist et non is_in_bonus_list(current, already_seen) alors
                    closest_bonus <- current
                    best_dist <- dist(current->b.x, current->b.y, runner->c.x, runner->c.y)
                Fin Si
                current <- current->next // On passe au bonus suivant
            Fin Tant que

            Retourner closest_bonus // On retourne le bonus le plus proche
        Fin
    \end{minted}
    \caption{Pseudo-code de la fonction \texttt{get\_closest\_bonus}.}
    \label{listing:c-get_closest_bonus}
\end{listing}

\newpage

\section{Module \texttt{combat\_moves}}
\label{sec:move-to-combat}

\begin{listing}[!htpb]
    \begin{minted}{c}
        void combat_moves(character_list runner, character_list closest_enemy, int* move_to_combat, levelinfo level);
    \end{minted}
    \caption{Prototype de \texttt{combat\_moves} en C.}
    \label{listing:c-combat_moves-prototype}
\end{listing}

\subsection{Description}

Si A* ne trouve aucun chemin pour aucun bonus, le mode de mouvement spécial est activé.
Dans ce mode, le runner va essayer de se rapprocher du bonus le plus proche, tout en évitant les ennemis.
Pour savoir comment se déplacer, si un ennemi est dangereux, on utilise la procédure \texttt{combat\_moves}.

\subsection{Paramètres}

\begin{table}[!htpb]
    \begin{tabularx}{\textwidth}{lXX}
        \toprule
        \textbf{Paramètre} & \textbf{Type} & \textbf{Description} \\
        \midrule
        \texttt{runner} & \texttt{character\_list} & Runner \\
        \texttt{closest\_enemy} & \texttt{character\_list} & Ennemi le plus proche. \\
        \texttt{move\_to\_combat} & \texttt{int*} & Pointeur vers l'action à effectuer. \\
        \texttt{level} & \texttt{levelinfo} & Structure contenant les informations du niveau. \\
        \bottomrule
    \end{tabularx}
    \caption{Paramètres de la procédure \texttt{combat\_moves}.}
    \label{tab:parameters-combat_moves}
\end{table}

\subsection{Choix d'Implémentation}

La procédure étant assez complexe, elle n'a pas été implémentée en une seule fois, mais petit à petit.
Cela vient du fait que les mouvements spéciaux sont assez complexes, et qu'il est difficile de tout prévoir dès le début: il faut tester et ajuster.
J'ai tout de même essayé de faire en sorte que la procédure soit la plus lisible possible, en utilisant des commentaires et des variables explicites.
\newline\newline
On utilise donc des Si / Sinon pour tester les différentes conditions.
De plus, on utilise une procédure car on modifie le pointeur \texttt{move\_to\_combat} en fonction de sa valeur actuelle.

\newpage

\subsection{Pseudo-code}

\begin{longlisting}
    \begin{minted}{text}
        Procéduer combat_moves
        Parametres :
            runner en character_list
            closest_enemy en character_list
            @move_to_combat en entier 
            level en levelinfo
        Declarations :
            down_left en caractère
            down_right en caractère
            top_left en caractère
            top_right en caractère
            left en caractère
            right en caractère
            center en caractère
            can_right en booléen
            can_left en booléen
            distance en entier
            can_up en booléen
            can_down en booléen
        Debut
            // On récupère les cases autour du runner
            down_left <- level.map[runner->c.y + 1][runner->c.x - 1]
            down_right <- level.map[runner->c.y + 1][runner->c.x + 1]
            top_left <- level.map[runner->c.y - 1][runner->c.x - 1]
            top_right <- level.map[runner->c.y - 1][runner->c.x + 1]
            left <- level.map[runner->c.y][runner->c.x - 1]
            right <- level.map[runner->c.y][runner->c.x + 1]
            center <- level.map[runner->c.y][runner->c.x]

            // On vérifie si on peut se déplacer à droite ou à gauche, et si on ne tombe pas en le faisant
            can_right <- is_valid(runner->c.y * level.xsize + runner->c.x, RIGHT, level, level)
            can_left <- is_valid(runner->c.y * level.xsize + runner->c.x, LEFT, level, level)
            can_right <- can_right et (down_right != BOMB et down_right != PATH)
            can_left <- can_left et (down_left != BOMB et down_left != PATH)

            Si closest_enemy != NULL alors
                // Si il y a un ennemi dangereux (donc que l'on est en mode combat)
                Si *move_to_combat == -1 alors
                    // Si on n'a pas encore décidé de comment se déplacer
                    distance <- runner->c.y - closest_enemy->c.y // On calcule la distance verticale
                    Si distance == 0 alors
                        // Combat horizontal
                        distance <- runner->c.x - closest_enemy->c.x // On calcule la distance horizontale
                        Si level.map[runner->c.y - 1][runner->c.x] == CABLE et level.map[runner->c.y + 1][runner->c.x] == PATH alors
                            // On est sur un cable, on ne peut pas poser de bombe, on saute
                            *move_to_combat <- DOWN
                        Sinon Si distance > 0 et distance < 4 alors
                            // A gauche
                            Si (down_left == FLOOR ou down_left == BOMB) et top_left != CABLE et left != ENEMY alors
                                Si down_left == BOMB alors
                                    // Il y a deja une bombe, on attend
                                    *move_to_combat <- NONE
                                Sinon
                                    // On pose une bombe a gauche
                                    *move_to_combat <- BOMB_LEFT
                                Fin Si
                            Sinon
                                // On ne peut pas poser de bombe, on se deplace a droite
                                Si can_right alors *move_to_combat <- RIGHT
                            Fin Si
                        Sinon Si distance < 0 et distance > -4 alors
                            // A droite
                            Si (down_right == FLOOR ou down_right == BOMB) et top_right != CABLE et right != ENEMY alors
                                Si down_right == BOMB alors
                                    // Il y a deja une bombe, on attend
                                    *move_to_combat <- NONE
                                Sinon
                                    // On pose une bombe a droite
                                    *move_to_combat <- BOMB_RIGHT
                                Fin Si
                            Sinon
                                // On ne peut pas poser de bombe, on se deplace a gauche
                                Si can_left alors *move_to_combat <- LEFT
                            Fin Si
                        Fin Si

                        // Ces mouvements ont priorite sur les autres

                        Si can_right alors
                            Si (left == LADDER ou center == LADDER ou (left == ENEMY et top_left == LADDER)) et distance > 0 alors
                                // On se deplace a droite si on est sur une echelle, ou si on a une echelle a gauche
                                *move_to_combat <- RIGHT
                            Fin Si
                        Fin Si

                        Si can_left alors
                            Si (right == LADDER ou center == LADDER ou (right == ENEMY et top_right == LADDER)) et distance < 0 alors
                                // On se deplace a gauche si on est sur une echelle, ou si on a une echelle a droite
                                *move_to_combat <- LEFT
                            Fin Si
                        Fin Si
                    Sinon
                        // Combat vertical (sur une echelle)
                        bool can_up <- is_valid(runner->c.y * level.xsize + runner->c.x, UP, level, level)
                        bool can_down <- is_valid(runner->c.y * level.xsize + runner->c.x, DOWN, level, level)
                        Si distance > 0 et can_down alors
                            // On descend si un ennemi est au dessus
                            *move_to_combat <- DOWN
                        Sinon Si distance < 0 et can_up alors
                            // On monte si un ennemi est en dessous
                            *move_to_combat <- UP
                        Fin Si

                        Si level.map[runner->c.y][runner->c.x] == PATH ou level.map[runner->c.y + 1][runner->c.x] == FLOOR alors
                            // Si on est en haut ou en bas d'une echelle, alors c'est une erreur et on est en pas en mode combat, on remmet move_to_combat a -1
                            *move_to_combat <- -1
                        Fin Si
                    Fin Si
                Fin Si
            Fin Si
        Fin
    \end{minted}
    \caption{Pseudo-code de la procédure \texttt{combat\_moves}.}
    \label{listing:c-combat_moves}
\end{longlisting}

\newpage

\section{Module \texttt{lode\_runner}}

\begin{listing}[!htpb]
    \begin{minted}{c}
        action lode_runner(levelinfo level, character_list characterl, bonus_list bonusl, bomb_list bombl);
    \end{minted}
    \caption{Prototype de \texttt{lode\_runner} en C.}
    \label{listing:c-lode_runner-prototype}
\end{listing}

\subsection{Description}

La fonction \texttt{lode\_runner} est la fonction principale de notre programme, c'est elle qui va appeler toutes les autres fonctions pour trouver le meilleur chemin pour le runner.
Elle renvoie l'action à effectuer par le runner.

\subsection{Paramètres}

\begin{table}[!htpb]
    \begin{tabularx}{\textwidth}{lXX}
        \toprule
        \textbf{Paramètre} & \textbf{Type} & \textbf{Description} \\
        \midrule
        \texttt{level} & \texttt{levelinfo} & Structure contenant les informations du niveau. \\
        \texttt{characterl} & \texttt{character\_list} & Liste des personnages. \\
        \texttt{bonusl} & \texttt{bonus\_list} & Liste des bonus. \\
        \texttt{bombl} & \texttt{bomb\_list} & Liste des bombes. \\
        \bottomrule
    \end{tabularx}
    \caption{Paramètres de la fonction \texttt{lode\_runner}.}
    \label{tab:parameters-lode_runner}
\end{table}

\subsection{Choix d'Implémentation}

Le point principal de cette fonction est l'itération sur les bonus.
Elle est donc centrée autour d'une boucle tant que (les bonus sont des listes chaînées).
On utilise des variables pour stocker les actions à effectuer, et on les retourne à la fin de la fonction.

\newpage

\subsection{Pseudo-code}

\begin{longlisting}
    \begin{minted}{text}
        Fonction lode_runner en action
        Parametres :
            level en levelinfo
            characterl en character_list
            bonusl en bonus_list
            bombl en bomb_list
        Declarations :
            runner en character_list
            astar_level en levelinfo
            already_seen en bonus_list
            closest_bonus en bonus_list
            to_exit en booléen
            move_to_combat en entier
            move_to_closest en entier
            move_to_path en entier
            move_to_skipped en entier
            @pat en path
            @c en child
            tmp en bonus_list
            v en entier
        Debut
            runner <- get_runner(characterl) // On récupère le runner
            level <- add_enemies(level, characterl, bombl) // On ajoute les ennemis à la carte
            // On créé un niveau pour A* avec des zones autour des ennemis. Le but est que le runner ne s'approche pas trop des ennemis
            astar_level <- get_astar_level(level, characterl) 

            already_seen <- NULL // On initialise la liste des bonus déjà vus
            closest_bonus <- get_closest_bonus(bonusl, runner, already_seen) // On récupère le bonus le plus proche

            Si bonusl == NULL alors
                // Si la liste des bonus est vide, c'est qu'on les a tous récupérés, on va donc vers la sortie
                // On crée un bonus fictif qui possède les coordonnées de la sortie
                closest_bonus <- malloc(sizeof(bonus_list))
                bonus b <- {level.xexit, level.yexit}
                closest_bonus->b <- b
                to_exit <- VRAI
            Fin Si

            // Ces variables vont stocker les actions à effectuer, elles sont initialisées à -1 pour qu'on sache si elles ont été modifiées
            move_to_combat <- -1
            move_to_closest <- -1
            move_to_path <- -1
            move_to_skipped <- -1

            Tant que closest_bonus != NULL faire
                // On itère sur les bonus
                pat <- a_star(runner, closest_bonus, astar_level, level) // On calcule le chemin vers le bonus le plus proche
                Si level.map[closest_bonus->b.y][closest_bonus->b.x] == ENEMY alors
                    // Cas spécial : si un ennemi est sur le bonus, on ne peut pas y aller (A* ne trouvera pas de chemin)
                    int runner_pos <- runner->c.y * level.xsize + runner->c.x
                    int closest_bonus_pos <- closest_bonus->b.y * level.xsize + closest_bonus->b.x
                    c <- find_closest_child(pat->p, runner_pos, closest_bonus_pos, level) // On trouve le chemin qui nous rapproche le plus du bonus
                    move_to_skipped <- get_action(runner_pos, c->pos, level) // On stocke l'action à effectuer
                    // On ajoute le bonus à la liste des bonus déjà vus
                    tmp <- malloc(sizeof(bonus_list))
                    tmp->b <- closest_bonus->b
                    tmp->next <- already_seen
                    already_seen <- tmp
                    Si non to_exit alors
                        // Si il y a d'autres bonus, on récupère le plus proche pour continuer la boucle
                        closest_bonus <- get_closest_bonus(bonusl, runner, already_seen)
                    Fin Si
                    Continuer
                Fin Si

                Si move_to_closest == -1 alors
                    // Si on n'a toujours initialisé move_to_closest, on le fait
                    // move_to_closest est l'action qui nous rapproche du bonus le plus proche (même s'il est inaccessible)
                    int runner_pos <- runner->c.y * level.xsize + runner->c.x
                    int closest_bonus_pos <- closest_bonus->b.y * level.xsize + closest_bonus->b

                    c <- find_closest_child(pat->p, runner_pos, closest_bonus_pos, level) // On trouve le chemin qui nous rapproche le plus du bonus
                    move_to_closest <- get_action(runner_pos, c->pos, level) // On stocke l'action à effectuer
                Fin Si

                Si pat->found alors
                    // Si on a trouvé un chemin, on le suit
                    v <- closest_bonus->b.y * level.xsize + closest_bonus->b.x
                    Tant que pat->p[v] != runner->c.y * level.xsize + runner->c.x faire
                        // pat->p[v] est le parent de v, on remonte le chemin pour trouver l'action à effectuer
                        v <- pat->p[v]
                    Fin Tant que

                    move_to_path <- get_action(runner->c.y * level.xsize + runner->c.x, v, level) // On stocke l'action à effectuer

                Sinon Si pat->heap->size != 0 alors
                    // Si le chemin est plus long que la taille du tas, c'est une erreur
                    Afficher "ERROR: Path is longer than heap size"
                    Sortir du programme
                Sinon
                    // Si on n'a pas trouvé de chemin, on est en mode combat
                    combat_moves(runner, get_closest_enemy(characterl, runner, level), &move_to_combat, level)
                Fin Si

                // On ajoute le bonus à la liste des bonus déjà vus
                tmp <- malloc(sizeof(bonus_list))
                tmp->b <- closest_bonus->b
                tmp->next <- already_seen
                already_seen <- tmp
                Si non to_exit alors
                    // Si il y a d'autres bonus, on récupère le plus proche pour continuer la boucle
                    closest_bonus <- get_closest_bonus(bonusl, runner, already_seen)
                Sinon
                    closest_bonus <- NULL
                Fin Si
            Fin Tant que

            Si move_to_path != -1 alors
                // On a trouvé un chemin, on le suit
                Retourner move_to_path
            Sinon Si move_to_combat != -1 alors
                // On est en mode combat
                Retourner move_to_combat
            Sinon Si move_to_closest != -1 alors
                // On n'a pas trouvé de chemin, on se rapproche du bonus
                Si is_valid_closest(runner->c.y * level.xsize + runner->c.x, move_to_closest, astar_level) alors
                    Retourner move_to_closest
                Sinon
                    Retourner NONE
                Fin Si
            Sinon
                // Les bonus sont inaccessibles, on cherche le chemin qui nous rapproche le plus d'un bonus
                Retourner move_to_skipped
            Fin Si
        Fin
    \end{minted}
    \caption{Pseudo-code de la fonction \texttt{lode\_runner}.}
    \label{listing:c-lode_runner}
\end{longlisting}