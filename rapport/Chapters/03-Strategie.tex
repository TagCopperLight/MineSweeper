\chapter{Stratégie}
\label{cp:strategie}

La stratégie utilisée pour terminer le jeu est construite autour de l'algorithme A* (\autoref{sec:a-star}).
\newline
On commence par récupérer la liste des bonus par ordre croissant de distance, et on essaie, en utilisant A*, de trouver un chemin pour atteindre le bonus le plus proche.
\newline
Si on trouve un chemin vers un des bonus, le runner le prendra : A* a une priorité sur toutes les autres décisions.
\newline
Si l'algorithme ne trouve aucun chemin vers aucun bonus (ce qui devient plus probable à mesure que le nombre d'ennemis augmente), le runner utilise des fonctions de mouvements dits "spéciaux" pour se déplacer.
\newline\newline
Ceci signifie que A* doit prendre en compte les ennemis, mais il y a plusieurs avantages à utiliser cette stratégie :

\begin{enumerate}
    \item On fait une confiance totale à A* : s'il sait qu'un ennemi n'est pas dangereux, il n'y a pas besoin de faire de mouvements pour l'éviter.
    \item On ne fait pas de mouvements inutiles : si A* trouve un chemin, c'est le chemin le plus court.
    \item Si A* ne trouve pas de chemin, on est sûr qu'il n'y en a pas, et on peut donc se permettre de faire des mouvements "spéciaux".
\end{enumerate}
Néanmoins, il y a un inconvénient : on ne peut pas prédire l'évolution de la position des ennemis.
Autrement dit les choix de A* se basent sur la situation à l'instant $t$.
\newline
C'est pour cela qu'on l'appelle à chaque tour, pour prendre en compte les nouvelles positions des ennemis.
\newline\newline
Heureusement, contrairement à l'imprécision de la position des ennemis qui augmente avec la distance, la dangerosité des ennemis diminue.
L'algorithme peut être à peu près sûr de la position des ennemis proches (les ennemis dangereux), et donc de les éviter.
\newpage
De plus, au lieu de passer la carte à A*, on lui passe une carte modifiée, dans laquelle les ennemis sont ajoutés, ainsi qu'une zone d'ennemis autour de chaque ennemi.
\newline
Cela permet d'éviter que le runner ne se rapproche à moins d'une case d'un ennemi, et ainsi de s'assurer que, même si le prochain tour est celui des ennemis, le runner ne sera pas en danger.

\begin{figure}[!htpb]
    \centering
    \includegraphics[width=0.75\linewidth]{Figures/diagramme2.png}
    \caption[Logigramme de la stratégie.]{Logigramme de la stratégie.}
    \label{fig:logigramme}
\end{figure}

\newpage

\section{Mouvements spéciaux}

Les mouvements spéciaux sont les mouvements effectués quand A* ne trouve aucun chemin.
Ils tentent, sans garantie de succès, d'améliorer la situation pour que A* puisse trouver un chemin lors d'un prochain tour.
\newline
Ils sont, contrairement à A*, beaucoup plus spécifiques.

\subsection{Mouvement de combat}

Comme vu dans le logigramme (\autoref{fig:logigramme}), un mouvement de combat est enregistré si A* ne trouve aucun chemin, qu'un ennemi est considéré comme dangereux et qu'il n'existe pas déjà un mouvement de combat.
Alors, on va utiliser une suite de conditions (\autoref{sec:move-to-combat}) pour déterminer l'action à réaliser (se déplacer, poser une bombe, attendre).

\subsection{Mouvement de rapprochement}

Si A* ne trouve aucun chemin et qu'aucun ennemi n'est considéré comme dangereux, c'est qu'un ou plusieurs ennemis bloquent le chemin.
On va alors essayer de se rapprocher du bonus, jusqu'à ce que : 
\begin{itemize}
    \item A* trouve un chemin.
    \item On se rapproche assez d'un ennemi pour qu'il soit considéré comme dangereux.
\end{itemize}