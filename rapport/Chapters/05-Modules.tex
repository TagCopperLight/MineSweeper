\chapter{Modules}
\label{chap:modules}

La stratégie se décompose en plusieurs modules, chacun ayant un rôle spécifique.
Ces modules sont indépendants les uns des autres et peuvent être utilisés individuellement.

\section{Liste des modules}

\subsection{Démineur}
\begin{table}[!htpb]
    \label{tab:modules-minesweeper}
    \begin{tabularx}{\textwidth}{lX}
        \toprule
        \textbf{Prototype} & \textbf{Description} \\
        \midrule
        \texttt{void initialize\_board()} & Initialise le plateau de jeu avec des cellules vides. \\
        \texttt{void place\_mines()} & Place les mines aléatoirement sur le plateau de jeu. \\
        \texttt{void calculate\_adjacent\_mines()} & Calcule le nombre de mines adjacentes pour chaque cellule. \\
        \texttt{bool reveal\_cell()} & Révèle une cellule et retourne vrai si la cellule ne contient pas de mine. \\
        \texttt{void flag\_cell()} & Marque une cellule comme contenant potentiellement une mine. \\
        \texttt{bool isWon()} & Vérifie si toutes les mines ont été correctement marquées et toutes les autres cellules révélées. \\
        \texttt{bool isLost()} & Vérifie si une mine a été révélée, ce qui entraîne la perte de la partie. \\
        \texttt{void print\_board()} & Affiche le plateau de jeu à l'écran. \\
        \bottomrule
    \end{tabularx}
    \caption{Fonctions du jeu de démineur.}
\end{table}

