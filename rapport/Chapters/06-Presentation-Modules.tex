\chapter{Présentation des modules}
\label{chap:presentation-modules}

\section{Module \texttt{initialize\_board}}

\begin{listing}[!htpb]
    \begin{minted}{c}
        Cell* initialize_board(int n, int m);
    \end{minted}
    \caption{Prototype de \texttt{initialize\_board} en C.}
    \label{listing:c-initialize_board-prototype}
\end{listing}

\subsection{Description}

Cette fonction initialise le plateau de jeu avec des cellules vides. Elle alloue de la mémoire pour un tableau 3D de cellules et établit les liens entre les cellules adjacentes.
\newline\newline
\subsection{Paramètres}

\begin{table}[!htpb]
    \label{tab:parameters-initialize_board}
    \begin{tabularx}{\textwidth}{lXX}
        \toprule
        \textbf{Paramètre} & \textbf{Type} & \textbf{Description} \\
        \midrule
        \texttt{n} & \texttt{int} & Nombre de lignes du plateau. \\
        \texttt{m} & \texttt{int} & Nombre de colonnes du plateau. \\
        \bottomrule
    \end{tabularx}
    \caption{Paramètres de la fonction \texttt{initialize\_board}.}
\end{table}

\subsection{Choix d'Implémentation}
Nous avons choisi d'implémenter cette fonction pour créer dynamiquement un plateau de jeu de taille variable. 
Cela permet de gérer facilement les cellules et leurs relations adjacentes, ce qui est essentiel pour le fonctionnement du jeu de démineur.

\subsection{Pseudo-code}
\begin{longlisting}
    \begin{minted}{text}
fonction initialize_board(n, m):
    allouer de la mémoire pour un tableau 3D de cellules appelé board
    pour i de 0 à n-1:
        allouer de la mémoire pour board[i]
        pour j de 0 à m-1:
            allouer de la mémoire pour board[i][j]
            initialiser board[i][j] avec :
                x = i
                y = j
                adjacentMines = 0
                isMine = faux
                isRevealed = faux
                isFlagged = faux
                allouer de la mémoire pour board[i][j].adjacentCells avec 8 éléments
                pour k de 0 à 7:
                    board[i][j].adjacentCells[k] = NULL

    pour i de 0 à n-1:
        pour j de 0 à m-1:
            si i-1 >= 0 et j-1 >= 0:
                board[i][j].adjacentCells[0] = board[i-1][j-1]
            si i-1 >= 0:
                board[i][j].adjacentCells[1] = board[i-1][j]
            si i-1 >= 0 et j+1 < m:
                board[i][j].adjacentCells[2] = board[i-1][j+1]
            si j+1 < m:
                board[i][j].adjacentCells[4] = board[i][j+1]
            si i+1 < n et j+1 < m:
                board[i][j].adjacentCells[7] = board[i+1][j+1]
            si i+1 < n:
                board[i][j].adjacentCells[6] = board[i+1][j]
            si i+1 < n et j-1 >= 0:
                board[i][j].adjacentCells[5] = board[i+1][j-1]
            si j-1 >= 0:
                board[i][j].adjacentCells[3] = board[i][j-1]

    origin = board[0][0]
    libérer la mémoire pour le tableau 3D board
    retourner origin
    \end{minted}
    \caption{Pseudo-code de la fonction \texttt{initialize\_board}.}
\end{longlisting}

\newpage

\section{Module \texttt{place\_mines}}

\begin{listing}[!htpb]
    \begin{minted}{c}
        void place_mines(Cell* origin, int numMines, int n, int m, int init_x, int init_y);
    \end{minted}
    \caption{Prototype de \texttt{place\_mines} en C.}
    \label{listing:c-place_mines-prototype}
\end{listing}

\subsection{Description}

Cette fonction place les mines aléatoirement sur le plateau de jeu, en évitant la zone initiale autour de la première cellule révélée.



\subsection{Choix d'Implémentation}

Nous avons choisi d'implémenter cette fonction pour garantir que les mines soient placées de manière aléatoire tout en évitant la zone initiale autour de la première cellule révélée. 
Cela permet de donner au joueur une chance équitable de commencer le jeu sans perdre immédiatement.

\subsection{Pseudo-code}

\begin{listing}[!htpb]
    \begin{minted}{text}
fonction place_mines(origin, numMines, n, m, init_x, init_y):
    minesPlaced = 0
    tant que minesPlaced < numMines:
        x = nombre aléatoire entre 0 et m-1
        y = nombre aléatoire entre 0 et n-1

        si init_x - 1 <= x <= init_x + 1 et init_y - 1 <= y <= init_y + 1:
            continuer // Éviter la zone initiale

        cell = origin
        pour i de 0 à x-1:
            cell = cell.adjacentCells[6]
        pour j de 0 à y-1:
            cell = cell.adjacentCells[4]

        si cell.isMine == faux:
            cell.isMine = vrai
            pour chaque cellule adjacente dans cell.adjacentCells:
                si cellule adjacente != NULL:
                    cellule adjacente.adjacentMines += 1
            minesPlaced += 1
    fin tant que
    \end{minted}
    \caption{Pseudo-code de la fonction \texttt{place\_mines}.}
\end{listing}

\newpage

\section{Module \texttt{print\_board}}
\section{Affichage récursif d'une grille de démineur}
\label{sec:affichage-recursif}

Dans un jeu de type démineur, l'affichage des cellules peut se faire de manière récursive. Lorsqu'une cellule est révélée, si elle ne contient pas de mines adjacentes, toutes ses cellules adjacentes sont également révélées. 
Ce processus se poursuit jusqu'à ce que toutes les cellules adjacentes sans mines soient révélées.

\begin{listing}[!htpb]
    \begin{minted}{c}
        void print_board(Cell* origin, int n, int m);
    \end{minted}
    \caption{Prototype de \texttt{print\_board} en C.}
    \label{listing:c-print_board-prototype}
\end{listing}

\subsection{Description}

Cette fonction affiche le plateau de jeu à l'écran, en montrant les cellules révélées, les mines, et les drapeaux.

\subsection{Paramètres}

\begin{table}[!htpb]
    \label{tab:parameters-print_board}
    \begin{tabularx}{\textwidth}{lXX}
        \toprule
        \textbf{Paramètre} & \textbf{Type} & \textbf{Description} \\
        \midrule
        \texttt{origin} & \texttt{Cell*} & Pointeur vers la cellule d'origine du plateau. \\
        \texttt{n} & \texttt{int} & Nombre de lignes du plateau. \\
        \texttt{m} & \texttt{int} & Nombre de colonnes du plateau. \\
        \bottomrule
    \end{tabularx}
    \caption{Paramètres de la fonction \texttt{print\_board}.}
\end{table}

\subsection{Choix d'Implémentation}

Nous avons choisi d'implémenter cette fonction pour permettre au joueur de visualiser l'état actuel du plateau de jeu. 
Cela inclut les cellules révélées, les mines et les drapeaux, ce qui est essentiel pour la prise de décision stratégique pendant le jeu.

\subsection{Pseudo-code}

\begin{longlisting}
    \begin{minted}{text}
fonction print_board(origin, n, m):
    imprimer "  " suivi des numéros de colonnes de 0 à m-1
    imprimer une nouvelle ligne

    rowStart = origin
    pour i de 0 à n-1:
        imprimer i suivi d'un espace
        cell = rowStart
        pour j de 0 à m-1:
            si cell.isRevealed:
                si cell.isMine:
                    imprimer "M "
                sinon:
                    imprimer cell.adjacentMines suivi d'un espace
            sinon si cell.isFlagged:
                imprimer "F "
            sinon:
                imprimer ". "
            cell = cell.adjacentCells[4]  // Aller à la cellule suivante à droite
        imprimer une nouvelle ligne
        rowStart = rowStart.adjacentCells[6]  // Aller à la ligne suivante en bas
    \end{minted}
    \caption{Pseudo-code de la fonction \texttt{print\_board}.}
\end{longlisting}

\section{Module \texttt{calculate\_adjacent\_mines}}


\begin{listing}[!htpb]
    \begin{minted}{c}
        void calculate_adjacent_mines(Cell* board, int n, int m);
    \end{minted}
    \caption{Prototype de \texttt{calculate\_adjacent\_mines} en C.}
    \label{listing:c-calculate_adjacent_mines-prototype}
\end{listing}

\subsection{Description}

Cette fonction calcule le nombre de mines adjacentes pour chaque cellule du plateau de jeu.

\subsection{Paramètres}

\begin{table}[!htpb]
    \label{tab:parameters-calculate_adjacent_mines}
    \begin{tabularx}{\textwidth}{lXX}
        \toprule
        \textbf{Paramètre} & \textbf{Type} & \textbf{Description} \\
        \midrule
        \texttt{board} & \texttt{Cell*} & Pointeur vers la cellule d'origine du plateau. \\
        \texttt{n} & \texttt{int} & Nombre de lignes du plateau. \\
        \texttt{m} & \texttt{int} & Nombre de colonnes du plateau. \\
        \bottomrule
    \end{tabularx}
    \caption{Paramètres de la fonction \texttt{calculate\_adjacent\_mines}.}
\end{table}

\subsection{Choix d'Implémentation}
Nous avons choisi d'implémenter cette fonction pour déterminer le nombre de mines adjacentes à chaque cellule. Cette information est cruciale pour le joueur afin de prendre des décisions éclairées sur les cellules à révéler ou à marquer.

\subsection{Pseudo-code}

\begin{longlisting}
    \begin{minted}{text}
fonction calculate_adjacent_mines(board, n, m):
    pour i de 0 à n-1:
        pour j de 0 à m-1:
            cell = board[i][j]
            cell.adjacentMines = 0
            pour chaque cellule adjacente dans cell.adjacentCells:
                si cellule adjacente != NULL et cellule adjacente.isMine:
                    cell.adjacentMines += 1
    \end{minted}
    \caption{Pseudo-code de la fonction \texttt{calculate\_adjacent\_mines}.}
\end{longlisting}

\section{Module \texttt{reveal\_cell}}

\begin{listing}[!htpb]
    \begin{minted}{c}
        bool reveal_cell(Cell* board, int x, int y, int n, int m);
    \end{minted}
    \caption{Prototype de \texttt{reveal\_cell} en C.}
    \label{listing:c-reveal_cell-prototype}
\end{listing}

\subsection{Description}

Cette fonction révèle une cellule et retourne vrai si la cellule ne contient pas de mine.

\subsection{Paramètres}

\begin{table}[!htpb]
    \label{tab:parameters-reveal_cell}
    \begin{tabularx}{\textwidth}{lXX}
        \toprule
        \textbf{Paramètre} & \textbf{Type} & \textbf{Description} \\
        \midrule
        \texttt{board} & \texttt{Cell*} & Pointeur vers la cellule d'origine du plateau. \\
        \texttt{x} & \texttt{int} & Coordonnée x de la cellule à révéler. \\
        \texttt{y} & \texttt{int} & Coordonnée y de la cellule à révéler. \\
        \texttt{n} & \texttt{int} & Nombre de lignes du plateau. \\
        \texttt{m} & \texttt{int} & Nombre de colonnes du plateau. \\
        \bottomrule
    \end{tabularx}
    \caption{Paramètres de la fonction \texttt{reveal\_cell}.}
\end{table}

\subsection{Choix d'Implémentation}

Nous avons choisi d'implémenter cette fonction pour permettre au joueur de révéler une cellule. 
Si la cellule ne contient pas de mine, elle est révélée et, si elle n'a pas de mines adjacentes, les cellules adjacentes sont également révélées récursivement.
\subsection{Pseudo-code}

\begin{longlisting}
    \begin{minted}{text}
fonction reveal_cell(board, x, y, n, m):
    cell = board
    pour i de 0 à x-1:
        cell = cell.adjacentCells[6]
    pour j de 0 à y-1:
        cell = cell.adjacentCells[4]

    si cell.isMine:
        retourner faux
    sinon:
        cell.isRevealed = vrai
        si cell.adjacentMines == 0:
            pour chaque cellule adjacente dans cell.adjacentCells:
                si cellule adjacente != NULL et non cellule adjacente.isRevealed:
                    reveal_cell(board, cellule adjacente.x, cellule adjacente.y, n, m)
        retourner vrai
    \end{minted}
    \caption{Pseudo-code de la fonction \texttt{reveal\_cell}.}
\end{longlisting}
\newpage

\section{Module \texttt{flag\_cell}}

\begin{listing}[!htpb]
    \begin{minted}{c}
        void flag_cell(Cell* board, int x, int y);
    \end{minted}
    \caption{Prototype de \texttt{flag\_cell} en C.}
    \label{listing:c-flag_cell-prototype}
\end{listing}

\subsection{Description}

Cette fonction marque une cellule comme contenant potentiellement une mine.

\subsection{Paramètres}

\begin{table}[!htpb]
    \label{tab:parameters-flag_cell}
    \begin{tabularx}{\textwidth}{lXX}
        \toprule
        \textbf{Paramètre} & \textbf{Type} & \textbf{Description} \\
        \midrule
        \texttt{board} & \texttt{Cell*} & Pointeur vers la cellule d'origine du plateau. \\
        \texttt{x} & \texttt{int} & Coordonnée x de la cellule à marquer. \\
        \texttt{y} & \texttt{int} & Coordonnée y de la cellule à marquer. \\
        \bottomrule
    \end{tabularx}
    \caption{Paramètres de la fonction \texttt{flag\_cell}.}
\end{table}

\subsection{Pseudo-code}

\begin{listing}[!htpb]
    \begin{minted}{text}
fonction flag_cell(board, x, y):
    cell = board
    pour i de 0 à x-1:
        cell = cell.adjacentCells[6]
    pour j de 0 à y-1:
        cell = cell.adjacentCells[4]

    cell.isFlagged = vrai
    \end{minted}
    \caption{Pseudo-code de la fonction \texttt{flag\_cell}.}
\end{listing}

\newpage

\section{Module \texttt{isLost}}

\begin{listing}[!htpb]
    \begin{minted}{c}
        bool isLost(Cell* board, int n, int m);
    \end{minted}
    \caption{Prototype de \texttt{isLost} en C.}
    \label{listing:c-isLost-prototype}
\end{listing}

\subsection{Description}

Cette fonction vérifie si une mine a été révélée, ce qui entraîne la perte de la partie.

\subsection{Paramètres}

\begin{table}[!htpb]
    \label{tab:parameters-isLost}
    \begin{tabularx}{\textwidth}{lXX}
        \toprule
        \textbf{Paramètre} & \textbf{Type} & \textbf{Description} \\
        \midrule
        \texttt{board} & \texttt{Cell*} & Pointeur vers la cellule d'origine du plateau. \\
        \texttt{n} & \texttt{int} & Nombre de lignes du plateau. \\
        \texttt{m} & \texttt{int} & Nombre de colonnes du plateau. \\
        \bottomrule
    \end{tabularx}
    \caption{Paramètres de la fonction \texttt{isLost}.}
\end{table}

\subsection{Pseudo-code}

\begin{listing}[!htpb]
    \begin{minted}{text}
fonction isLost(board, n, m):
    pour i de 0 à n-1:
        pour j de 0 à m-1:
            cell = board[i][j]
            si cell.isRevealed et cell.isMine:
                retourner vrai
    retourner faux
    \end{minted}
    \caption{Pseudo-code de la fonction \texttt{isLost}.}
\end{listing}